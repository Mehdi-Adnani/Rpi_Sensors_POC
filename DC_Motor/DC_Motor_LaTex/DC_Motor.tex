\documentclass{article}
\usepackage[utf8]{inputenc}
\usepackage{tikz}
\usepackage{circuitikz}
\usepackage{geometry}
\geometry{margin=1in}
\usepackage{listings}
\usepackage{xcolor}
\geometry{margin=1in}
\title{DC Motor Documentation - Raspberry Pi 5 Integration}
\author{El Mehdi Adnani Kadmiri}
\date{July 21, 2025}

\lstset{
	basicstyle=\ttfamily\small,
	keywordstyle=\color{blue},
	commentstyle=\color{gray},
	stringstyle=\color{teal},
	breaklines=true,
	frame=single,
	postbreak=\mbox{\textcolor{red}{$\hookrightarrow$}\space}
}

\begin{document}
	
	\maketitle
	
	\section{Description}
	A \textbf{DC Motor} (Direct Current Motor) is a simple electric motor that runs on direct current electricity. It converts electrical energy into mechanical motion. When voltage is applied to the motor terminals, it generates torque and rotation.
	
	DC motors are widely used for their simplicity, low cost, and ability to run at variable speeds. They can be controlled using a motor driver like the L298N or an H-bridge circuit that handles the higher current required by the motor.
	
	\section{Applications}
	\begin{itemize}
		\item \textbf{Robotic Wheels:} Used for locomotion in robotic platforms.
		\item \textbf{Conveyor Belts:} Employed in manufacturing and sorting systems.
		\item \textbf{Automated Mechanisms:} Useful in fans, pumps, and small appliances.
	\end{itemize}
	
	\section{Working Principle}
	A DC motor works based on the interaction between a magnetic field and a current-carrying conductor. When current flows through the armature inside the magnetic field, it experiences a force (Lorentz force) causing rotation.
	
	Controlling a DC motor from the Raspberry Pi usually requires:
	\begin{itemize}
		\item A motor driver like \textbf{L298N} to handle high current
		\item \textbf{PWM} signals to adjust speed
		\item GPIO logic to control direction
	\end{itemize}
	
	\section{Wiring Diagram}
	\begin{center}
		\begin{tabular}{|c|c|c|}
			\hline
			\textbf{L298N Pin} & \textbf{Raspberry Pi Pin} & \textbf{Function} \\
			\hline
			IN1 & GPIO17 & Direction Control \\
			IN2 & GPIO27 & Direction Control \\
			ENA & GPIO18 & PWM Speed Control \\
			VCC & External 12V & Motor Power \\
			GND & GND & Common Ground \\
			\hline
		\end{tabular}
	\end{center}
	
	\section{Libraries Used}
	\subsection*{Python: RPi.GPIO}
	\texttt{RPi.GPIO} provides control over GPIO pins and can generate PWM signals for speed control.
	
	\begin{itemize}
		\item Import: \texttt{import RPi.GPIO as GPIO}
		\item Setup: \texttt{GPIO.setmode(GPIO.BCM)}
		\item Set pin mode: \texttt{GPIO.setup(pin, GPIO.OUT)}
		\item PWM: \texttt{GPIO.PWM(pin, frequency)}
	\end{itemize}
	
	\subsection*{C: wiringPi + softPwm}
	\texttt{wiringPi} and \texttt{softPwm} provide GPIO and PWM functions in C.
	
	\begin{itemize}
		\item Setup: \texttt{wiringPiSetup()}
		\item Direction: \texttt{pinMode(pin, OUTPUT)}
		\item PWM: \texttt{softPwmCreate(pin, value, range)}
	\end{itemize}
	
	\section{Python Example}
	\begin{lstlisting}[language=Python]
		import RPi.GPIO as GPIO
		import time
		
		IN1 = 17
		IN2 = 27
		ENA = 18
		
		GPIO.setmode(GPIO.BCM)
		GPIO.setup(IN1, GPIO.OUT)
		GPIO.setup(IN2, GPIO.OUT)
		GPIO.setup(ENA, GPIO.OUT)
		
		pwm = GPIO.PWM(ENA, 1000)  # 1kHz PWM
		pwm.start(50)  # 50% speed
		
		try:
		GPIO.output(IN1, GPIO.HIGH)
		GPIO.output(IN2, GPIO.LOW)
		time.sleep(5)
		except KeyboardInterrupt:
		pass
		
		pwm.stop()
		GPIO.cleanup()
	\end{lstlisting}
	
	\section{C Example}
	\begin{lstlisting}[language=C]
		#include <wiringPi.h>
		#include <softPwm.h>
		#include <stdio.h>
		
		#define IN1 0 // GPIO17
		#define IN2 2 // GPIO27
		#define ENA 1 // GPIO18
		
		int main(void) {
			wiringPiSetup();
			
			pinMode(IN1, OUTPUT);
			pinMode(IN2, OUTPUT);
			softPwmCreate(ENA, 0, 100);
			
			digitalWrite(IN1, HIGH);
			digitalWrite(IN2, LOW);
			softPwmWrite(ENA, 50); // 50% speed
			
			delay(5000); // Run motor for 5 seconds
			
			softPwmWrite(ENA, 0);
			return 0;
		}
	\end{lstlisting}
	
	\section{Conclusion}
	The DC motor is a versatile component suitable for many motion-based applications. When combined with an L298N driver and GPIO/PWM control, it can be precisely managed from the Raspberry Pi using either Python or C.
\end{document}
