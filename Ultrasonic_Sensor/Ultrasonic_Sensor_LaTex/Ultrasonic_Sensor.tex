\documentclass{article}
\usepackage[utf8]{inputenc}
\usepackage{geometry}
\usepackage{amsmath}
\usepackage{listings}
\usepackage{xcolor}
\geometry{margin=1in}
\title{Ultrasonic Sensor Documentation - Raspberry Pi 5 Integration}
\author{El Mehdi Adnani Kadmiri}
\date{July 17, 2025}

\begin{document}
	
	\maketitle
	
	\section{Description}
	The Ultrasonic Sensor (e.g., HC-SR04) uses ultrasonic sound waves to measure the distance between the sensor and an object. It sends out a sound pulse at 40kHz and measures the time taken for the echo to return. This time is then used to calculate distance using the speed of sound.
	
	\section{Applications}
	\begin{itemize}
		\item Distance measurement
		\item Obstacle detection in robotics
		\item Liquid level sensing
		\item Smart parking and automation systems
	\end{itemize}
	
	\section{Working Principle}
	The sensor has two main pins:
	\begin{itemize}
		\item \textbf{Trigger (TRIG)}: Sends an ultrasonic burst (10µs pulse).
		\item \textbf{Echo (ECHO)}: Receives the reflected wave and outputs a pulse proportional to distance.
	\end{itemize}
	Using the time between sending and receiving, distance is calculated with:
	\[
	\text{Distance (cm)} = \frac{\text{Time (µs)} \times 0.0343}{2}
	\]
	
	\section{Wiring Diagram}
	\begin{center}
		\begin{tabular}{|c|c|c|}
			\hline
			\textbf{Ultrasonic Pin} & \textbf{Raspberry Pi Pin} & \textbf{Function} \\
			\hline
			VCC & 5V & Power \\
			GND & GND & Ground \\
			TRIG & GPIO23 & Output Trigger Signal \\
			ECHO & GPIO24 & Input Echo Signal \\
			\hline
		\end{tabular}
	\end{center}
	
	\section{Libraries Used}
	\subsection*{Python: RPi.GPIO}
	\begin{itemize}
		\item Setup with: \texttt{import RPi.GPIO as GPIO}, \texttt{import time}
		\item Configure mode: \texttt{GPIO.setmode(GPIO.BCM)}
		\item Trigger pulse: \texttt{GPIO.output(TRIG, True)}
		\item Read echo: \texttt{GPIO.input(ECHO)}
	\end{itemize}
	
	\subsection*{C: wiringPi}
	\begin{itemize}
		\item Initialize GPIO: \texttt{wiringPiSetupGpio();}
		\item Set pins: \texttt{pinMode(TRIG, OUTPUT)}, \texttt{pinMode(ECHO, INPUT)}
		\item Control and read: \texttt{digitalWrite()}, \texttt{digitalRead()}, \texttt{micros()}
	\end{itemize}
	
	\section{Python Code Example}
	\begin{lstlisting}[language=Python]
		import RPi.GPIO as GPIO
		import time
		
		TRIG = 23
		ECHO = 24
		
		GPIO.setmode(GPIO.BCM)
		GPIO.setup(TRIG, GPIO.OUT)
		GPIO.setup(ECHO, GPIO.IN)
		
		GPIO.output(TRIG, False)
		time.sleep(2)
		
		GPIO.output(TRIG, True)
		time.sleep(0.00001)
		GPIO.output(TRIG, False)
		
		while GPIO.input(ECHO) == 0:
		pulse_start = time.time()
		
		while GPIO.input(ECHO) == 1:
		pulse_end = time.time()
		
		pulse_duration = pulse_end - pulse_start
		distance = (pulse_duration * 34300) / 2
		
		print("Distance: %.2f cm" % distance)
		GPIO.cleanup()
	\end{lstlisting}
	
	\section{C Code Example}
	\begin{lstlisting}[language=C]
		#include <wiringPi.h>
		#include <stdio.h>
		#include <stdlib.h>
		#include <sys/time.h>
		
		#define TRIG 23
		#define ECHO 24
		
		long getMicroseconds() {
			struct timeval tv;
			gettimeofday(&tv, NULL);
			return tv.tv_sec * 1000000 + tv.tv_usec;
		}
		
		int main(void) {
			if (wiringPiSetupGpio() == -1)
			return 1;
			
			pinMode(TRIG, OUTPUT);
			pinMode(ECHO, INPUT);
			
			digitalWrite(TRIG, LOW);
			delay(500);
			
			digitalWrite(TRIG, HIGH);
			delayMicroseconds(10);
			digitalWrite(TRIG, LOW);
			
			while (digitalRead(ECHO) == LOW);
			long start = getMicroseconds();
			
			while (digitalRead(ECHO) == HIGH);
			long end = getMicroseconds();
			
			long duration = end - start;
			float distance = duration * 0.0343 / 2;
			
			printf("Distance: %.2f cm\n", distance);
			return 0;
		}
	\end{lstlisting}
	
	\section{Conclusion}
	The Ultrasonic Sensor provides accurate, non-contact distance measurement and is widely used in robotics. With basic GPIO and timing functions, it integrates well with the Raspberry Pi using both Python and C.
	
\end{document}
