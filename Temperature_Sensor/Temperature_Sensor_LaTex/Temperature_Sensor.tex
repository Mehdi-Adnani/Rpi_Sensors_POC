\documentclass{article}
\usepackage[utf8]{inputenc}
\usepackage{geometry}
\usepackage{amsmath}
\usepackage{listings}
\geometry{margin=1in}
\title{DHT Sensor Documentation - Raspberry Pi 5 Integration}
\author{El Mehdi Adnani Kadmiri}
\date{July 17, 2025}

\begin{document}
	
	\maketitle
	
	\section{Description}
	The DHT sensor (e.g., DHT11 or DHT22) is used for measuring temperature and humidity. It includes a thermistor and a capacitive humidity sensor to generate calibrated digital output.
	
	\section{Applications}
	\begin{itemize}
		\item Weather stations
		\item Home automation
		\item Greenhouse monitoring
		\item IoT environmental sensing
	\end{itemize}
	
	\section{Working Principle}
	The DHT sensor measures temperature and humidity and sends the data as a digital signal via a single data pin. The sensor requires specific timing to read the data signal, which is parsed into two values: temperature and humidity.
	
	\section{Wiring Diagram}
	\begin{center}
		\begin{tabular}{|c|c|c|}
			\hline
			\textbf{DHT Pin} & \textbf{Raspberry Pi Pin} & \textbf{Function} \\
			\hline
			VCC & 3.3V or 5V & Power \\
			GND & GND & Ground \\
			DATA & GPIO4 & Data Signal \\
			\hline
		\end{tabular}
	\end{center}
	
	\section{Libraries Used}
	\subsection*{Python: Adafruit\_DHT}
	This library simplifies interaction with DHT sensors.
	\begin{itemize}
		\item Install via pip: \texttt{pip install Adafruit-DHT}
		\item Usage: \texttt{humidity, temperature = Adafruit\_DHT.read\_retry(sensor, pin)}
	\end{itemize}
	
	\subsection*{C: WiringPi and Custom Protocol}
	There is no official WiringPi support; bit-banging and timing logic are used in custom C implementations.
	\begin{itemize}
		\item GPIO access: \texttt{wiringPiSetupGpio()}
		\item Signal parsing: Requires microsecond timing
	\end{itemize}
	
	\section{Python Code Example}
	\begin{lstlisting}[language=Python]
		import Adafruit_DHT
		
		sensor = Adafruit_DHT.DHT11
		pin = 4
		
		humidity, temperature = Adafruit_DHT.read_retry(sensor, pin)
		
		if humidity is not None and temperature is not None:
		print(f"Temp={temperature:.1f}C  Humidity={humidity:.1f}%")
		else:
		print("Sensor failure. Check wiring.")
	\end{lstlisting}
	
	\section{C Code Example}
	\begin{lstlisting}[language=C]
		#include <wiringPi.h>
		#include <stdio.h>
		#include <stdlib.h>
		#include <stdint.h>
		
		#define MAX_TIMINGS 85
		#define DHT_PIN 4
		
		int data[5] = {0,0,0,0,0};
		
		void read_dht_data() {
			uint8_t laststate = HIGH;
			uint8_t counter = 0;
			uint8_t j = 0, i;
			
			data[0] = data[1] = data[2] = data[3] = data[4] = 0;
			
			pinMode(DHT_PIN, OUTPUT);
			digitalWrite(DHT_PIN, LOW);
			delay(18);
			digitalWrite(DHT_PIN, HIGH);
			delayMicroseconds(40);
			pinMode(DHT_PIN, INPUT);
			
			for (i = 0; i < MAX_TIMINGS; i++) {
				counter = 0;
				while (digitalRead(DHT_PIN) == laststate) {
					counter++;
					delayMicroseconds(1);
					if (counter == 255)
					break;
				}
				laststate = digitalRead(DHT_PIN);
				if (counter == 255) break;
				if ((i >= 4) && (i % 2 == 0)) {
					data[j / 8] <<= 1;
					if (counter > 16) data[j / 8] |= 1;
					j++;
				}
			}
			
			if ((j >= 40) && (data[4] == ((data[0] + data[1] + data[2] + data[3]) & 0xFF))) {
				float humidity = data[0];
				float temperature = data[2];
				printf("Humidity = %.1f %% Temperature = %.1f *C\n", humidity, temperature);
			} else {
				printf("Data not good, skip\n");
			}
		}
		
		int main(void) {
			if (wiringPiSetupGpio() == -1)
			return 1;
			while (1) {
				read_dht_data();
				delay(2000);
			}
			return 0;
		}
	\end{lstlisting}
	
	\section{Conclusion}
	The DHT sensor offers a compact and easy-to-use solution for measuring environmental temperature and humidity, making it perfect for Raspberry Pi-based monitoring systems.
	
\end{document}